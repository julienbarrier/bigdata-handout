\chapter{Neural networks, from single neuron to multilayer networks}
\label{ch:neural-networks}
9 apr 18

most of the reason why people are excited about machine learning is
what has been made possible with neural networks. it is mostly related to what
we already seen.

representation of real neuron, VS mathematical object.
inspiration from biology.
brain organised as a network of neurons. each of these neurons are a processor
of information. many dendrites on which action potential coming in.
then axon, transfer of information as action potential.

scheme.
$y=\phi (u)$
$u = \sum_{j=1}^p \omega_j x_j$
$x_0 =1$

first example when $\phi$ is linear. linear regression.
example:
$\phi(u) = u$.
$l(\omega) = \sumin (y_i - \phi(u_i))^2$
taking the sum of squares. $\phi(u_i) = \sum_{j=1}^p \omega_j x_j$.
linear regression, first example.

another example that we've seen: logistic regresion (for classification).

in this context, we took:
$\phi(u) = \frac{1}{1+e^{-u}}$.
the cost function can be written as:
$l(\omega) = - \sumin k_i \ln y_i + (1-k_i) \ln (1-y_i)$
$y_i = P(k_i = 1|\omega,x) = \frac{e^{\omega^T x}}{1+e^{\omega^T x}}$
it is what we presented a few pages ago.
$l(\omega)$ looks very much like an entropy.
$y_i$ looks like a Boltzman distribution with a two-states system.

$P(E) = \frac{e^-{\betaE}}{Z}$: real boltzman distribution. two energies
E1 and E2. $P(E=E_1) = \frac{e^{\betaE}}{e^{-\betaE_0 + e^{-\beta E_1}}} = \frac{1}{1+e^{\beta\Delta E}}$.

if $K\geq 2$,
soft-max :
$k_{im} = 1$ if $i$ is in class $m$.
$i \in {1,\ldots, N}$ $m \in {1,\ldots,K}$.

generalisation of the same formula when we have more than two states:
$P(k_m = 1|x,\omega) = \frac{e^{x^T\omega_m}}{\sum_{n=1}^K e^{x^T\omega_m}}$

$l(\omega) = -\sumin \sum_{m=1}^K (k_{im} \ln y_{im} + (1-k_{im})\ln (1-y_{im}))$.

to find the best parameter of this model, write the function: maximum likelihood
estimation (MLE). write:

$\mathcal{L} = \ln P(\cdots)$
to minimise: $l = - \mathcal{L}$.

standard approach to define a model that is defined in term of probabilities.
mainly to remind that we've already seen many examples on single neurons.
essentially this will be discused later with classification of images.

network of neurons, the last layer will convert the likelihood into probabilities.
what is the probability that it is a dog, a cat, etc.

many different non linear functions that can be used. we can see the sigmoids.
nowadays in neural networks, ReLU and Leary ReLU are mostly used.

Leaky ReLU: does not saturate at high U

$\phi(u) = 1 if u>0, 0 if u \leq 0$.
let's consider $p=2$, $\omega_1 = \omega_2 = -2$, $\omega_0 = 3$.
let's consider that $x_1$ and $x_2$ are binary values : 0 or 1.
and then that $u = 3-2(x_1+x_2)$.

Let's see what are the different options.

$x_2$ \ $x_1$   0   1
0               1   1
1               1   0

it is a NAND gate
it is a way whith which we can make logical combinations.
this framework is one particular type of neurons. If I can do this, I can do any
kind of computation.
if we are doing picture recognition, input = pixels, output = what it is.


try to do some optimisation, try to organise the NAND gates, to get the circuit
that is going to achieve the function that we want. The big thing is that 
we need an algorithm to get the perfect circuit, network. with gradient descent
we can do it.
we don't have to think too much about the connection.
There would be two components.
the weights themselves are parameters.

If I take something more smooth by playing with the weights, I can do any
calculation that I want.

\subsection{Multilayer feedforward network}

in general, we could imagine many neural networks.
we have to train the network.
very simple algorithm: called multilayer feedforward network. network organised 
in layers.
last layer : output, classification in classes : number K of outputs.
input layer : pixels for example.
and inbetween, many hidden layers. feedforward, because all the connections are from one layer to the next, all oriented in the same way.

another type:

\subsection{Backpropagation algorithm}
smart implementation of gradient descent.
we can start with a single layer.
we've already seen this gradient descent algorithm.

single layer:
$l(\omega) = \half \sumin \sum_{m=1}^K (y_{im} - \phi(u_{im}))^2$
$u_{im} = \omega_m^T x_i = \sum_{j=1}^p \omega_{mj}x_{ij}$

$\frac{\partial l}{\partial \omega_{mj}} = - \sumin (y_{im} - \phi (u_{im}) \phi'(u_{im}) \frac{\partial u_{im}}{\partial \omega_{mj}}$
 = $ - \sumin e_{im} x_{ij}$

gradient descent

$\omega_{mj} \leftarrow \omega_{mj} - \eta \frac{\partial l}{\partial \omega_{mj}} = \omega_{mj} - \eta \sumin e_{im} x_{ij}$

on line: $\omega_{mj} \leftarrow \omega_{mj} - \eta e_{im} x_{ij}$.

we're gonna do the same in the context of neural networks.

$l(\omega) = \half \sumin \sum_{m=1}^k (y_{im} - x_{im}^{(L)})^2$

given any input, I update the data, layer by layer. that is called forward 
propagation : $x^(0) \rigtarrow x^{(1)} \rightarrow \cdots \rightarrow x^{(L)}$

$x_{j_k}^{(k)} = \phi (u_{j_k}^{(k)})$.
$u_{j_k}^{(k)} = \sum_{j_{k-1}} \omega_{j_k j_{k-1}} x_{j_{k-1}}^{(k-1)}$

secondly, bacward propagation:
$e^{(L)} \rightarrow e^{(L-1)} \rightarrow \cdots \rightarrow e^{(1)}$

$e_{j_L}^{(L)} = - (y_{j_L}^{(L)} - x_{j_L}^{(L)}) \phi'(u_{j_L}^{(L)}$


$e_{j_{k-1}}^{(k-1)} = \phi'(u_{j_{k-1}}^{(k-1)} \sum_{j_k} \omega_{j_kj_{k-1}}^{(k)} e_{j_k}^{(k)}$

At the end, when we've done this forward an d backward, we have an update, that
is the gradient descent. just take the product of the erors:

$\omega_{ab}^{(n)} \leftarrow \omega_{ab}^{(n)} - \eta e_a^{(n)} x_b^{(n-1)}$

What we'll try to do is derive??? those equations.

$\frac{\partial l}{\partial \omega_{ab}^{(n)}} = - \sum_{j_L = 1}^N (y_{j_L} - \phi (u_{j_L})) \phi'(u_{j_L}) \frac{\partial u_{j_L}}{\partial
\omega_{ab}^{(n)}}$

let's redefine the indexes:
layer k.
index j for the neuron inside the layer.
then we have $x_j^{(k)} = \phi(u_{j_k}^{(k)})$.

$\frac{\partial l}{\partial \omega_{ab}^{(n)}} = \sum_{j_{L-1}} \frac{\partial x_{j_{L-1}}^{(L-1)}}{\partial \omega_{ab}^{(n)}} \sum_{j_L}
\omega_{j_Lj_{L-1}}^{(L)} e_{j_L}^{(L)}$

this is true for $k = L-1$ as shown:

$\frac{\partial l}{\partial \omega_{ab}^{(n)}} = \sum_{j_k} \frac{x_{j_k}^{(k)}}{\partial \omega_{ab}^{(n)}} \sum_{j_{k+1}}
\omega_{j_{k+1}j_k}^{(k+1)}e_{j_{k+1}}^{(k+1)}$

with $\frac{\pratial x_{j_k}^{(k)}}{\partial \omega_{ab}^{(n)}} = \delta_{j_k}^a x_b^{(k)} \phi'(u_{j_k}^{(k)}$.
for n = k,

frac(\dots) = $x_b^{(k)} \phi'(u_a^{(k)} ) \sum_{j_{k+1}} \omega_{j_{k+1}a}^{(k+1)} e_{j_{k+1}}^{(k+1)}$


example. recognition of digits. coming from US postal services.
10 classes.
as inputs, we need one neuron for each pixel. output: we used something like
soft max. at the end, once we have trained these networks, set the value of the
first layer based on the digits. 

this example to show that only 15 neurons in the hidden layer.
what is written is only the backpropagation algorithm. no hyper parameter.
split into training set, and test set. training set larger.
let's look at the website, start a very complicated code.
relatively simple, already does 95\% accuracy. state of the art:99.8\%, that is 
what human would do.
deep neural networks: much many layers.

more complicated network can perform to the level of human perfection.
current application:
- image recognition picture, convolution networks
- process sequences that means process speech, movies or text. slightly
different architecture.
bit of idea of what goes in the current neural networks. essentially, the basic 
architecture\dots..

in one particular application, convolution networks, \dots
idea before going into description.
different layers, image is the first.
we want a transformation of these data in such a way that at the end, we can
apply a soft-max. \ie take our points, and put together all the images that are similar. there's a problem of the distance between the images. if we
take the raw images, take all the pictures of cats/dogs together.
at the end, project our data.
somewhat there are some special properties of images. if we make a slight 
rotation of the image, it doesn't change the distances.
identify the edges in the picture.
at the end, we convert the image in a series of activation maps, by applying
many filters. question: what filter should I use.
we are going to train the filters to find the good value of the weight that
is going to identify the particular data in the image.



mostly done with labels???


now, come back to PCA.
principal component analysis. simplest but most powerful approach that can be
applied to data.
that is unsupervised.

as usual ,the data are:
$x_{ij}$, $i = 1, \ldots, N$, $j=1,\ldots, p$.
what we have to compute is the covariance matrix:
$C_{ij}$. we normalise at first the data by substracting the mean.

$x'_{ij} = x_{ij} - \nth \sum_{a=1}^N x_{aj}$

$x' = x - \bar{x}$
then

$x'' = \frac{x'}{\sigma(x')}$

therefore the matrix C is p times p dimension.
So we will compute:
$C_{jk} = \nth \sumin x_{ij} x_{ik}$. it is a correlation function.
we can write it as:

$C_{jk} = \overline{x_j x_k}$.
in general, we should avoid making loops for sums, and find ways to compute it
with linear algearba. in this case:
$C_{jk} = \nth X^TX$
the eigenvalues and eigenvectors are :
$v_1, v_2, \ldots$ associated with $\lambda_1 \geq \lambda_2 \geq \cdots$.
In general, we have a matrix V composed of the eigenvetors in the columns.

we can represent the data by plotting the data versus $v_1$ and $v_2$.
they are defined as a multiplicative constant. and then project the data.
interpretation of that kind of diagram.
each feature is contributing to the eigenvectors. we have to think of them as
modes.
if we look back at the examples given by financial data, we are correlating the return of our stocks. When we have two companies that are behaving in
similar ways.









